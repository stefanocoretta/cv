%%%%%%%%%%%%%%%%%%%%%%%%%%%%%%%%%%%%%%%%%
% Developer CV
% LaTeX Template
% Version 1.0 (28/1/19)
%
% This template originates from:
% http://www.LaTeXTemplates.com
%
% Authors:
% Jan Vorisek (jan@vorisek.me)
% Based on a template by Jan Küster (info@jankuester.com)
% Modified for LaTeX Templates by Vel (vel@LaTeXTemplates.com)
%
% License:
% The MIT License (see included LICENSE file)
%
%%%%%%%%%%%%%%%%%%%%%%%%%%%%%%%%%%%%%%%%%

%----------------------------------------------------------------------------------------
%	PACKAGES AND OTHER DOCUMENT CONFIGURATIONS
%----------------------------------------------------------------------------------------

\documentclass[9pt]{developercv} % Default font size, values from 8-12pt are recommended

%----------------------------------------------------------------------------------------

\definecolor{darkred}{HTML}{C11B17}

\begin{document}

%----------------------------------------------------------------------------------------
%	TITLE AND CONTACT INFORMATION
%----------------------------------------------------------------------------------------

\begin{minipage}[t]{0.5\textwidth} % 45% of the page width for name
	\vspace{-\baselineskip} % Required for vertically aligning minipages

	% If your name is very short, use just one of the lines below
	% If your name is very long, reduce the font size or make the minipage wider and reduce the others proportionately
	\colorbox{darkred}{{\HUGE\textcolor{white}{\textbf{\MakeUppercase{Stefano}}}}} % First name

	\colorbox{darkred}{{\HUGE\textcolor{white}{\textbf{\MakeUppercase{Coretta}}}}} % Last name

	\vspace{6pt}

	{\huge Postdoc Researcher} % Career or current job title

	Ludwig--Maximilians--Universität München
\end{minipage}
\begin{minipage}[t]{0.5\textwidth} % 27.5% of the page width for the first row of icons
	\vspace{-\baselineskip} % Required for vertically aligning minipages

	% The first parameter is the FontAwesome icon name, the second is the box size and the third is the text
	% Other icons can be found by referring to fontawesome.pdf (supplied with the template) and using the word after \fa in the command for the icon you want
	\icon{MapMarker}{12}{Institute of Phonetics and Speech Processing (IPS)}\\
	\icon{At}{12}{\href{mailto:s.coretta@lmu.de}{s.coretta@lmu.de}}\\
  \icon{Globe}{12}{\href{https://stefanocoretta.github.io}{stefanocoretta.github.io}}\\
  \icon{Github}{12}{\href{https://github.com/stefanocoretta}{github.com/stefanocoretta}}\\
\end{minipage}

\vspace{0.25cm}

%----------------------------------------------------------------------------------------
%	INTRODUCTION, SKILLS AND TECHNOLOGIES
%----------------------------------------------------------------------------------------

% \cvsect{Who Am I?}

% \begin{minipage}[t]{0.4\textwidth} % 40% of the page width for the introduction text
% 	\vspace{-\baselineskip} % Required for vertically aligning minipages

% 	\lorem \lorem \lorem \lorem \lorem\\ % Dummy text
% \end{minipage}
% \hfill % Whitespace between
% \begin{minipage}[t]{0.5\textwidth} % 50% of the page for the skills bar chart
% 	\vspace{-\baselineskip} % Required for vertically aligning minipages
% 	\begin{barchart}{5.5}
% 		\baritem{JavaScript}{60}
% 		\baritem{PHP}{100}
% 		\baritem{SASS/LESS}{70}
% 		\baritem{Bootstrap}{70}
% 		\baritem{Git}{40}
% 		\baritem{LaTeX}{60}
% 	\end{barchart}
% \end{minipage}

% \begin{center}
% 	\bubbles{5/Eclipse, 6/git, 4/Office, 3/Inkscape, 3/Blender}
% \end{center}

%----------------------------------------------------------------------------------------
%	EXPERIENCE
%----------------------------------------------------------------------------------------

\cvsect{Employment}

\begin{entrylist}
	\entry
		{2019--present}
		{Postdoc Researcher}
		{Ludwig-Maximilians-Universität München, DE}
		{Institute of Phonetics and Speech Processing}
	\entry
		{2017--2018}
		{Phonetics Laboratory Assistant}
		{University of Manchester, UK}
		{}
	\entry
		{2015}
		{Junior Linguist}
		{Adecco at Google Italy, Milan}
		{}
\end{entrylist}

%----------------------------------------------------------------------------------------
%	EDUCATION
%----------------------------------------------------------------------------------------

\cvsect{Education}

\begin{entrylist}
	\entry
		{2016--2020}
		{PhD in Linguistics}
		{University of Manchester, UK}
		{}
	\entry
		{2015--2016}
		{MA in Phonetics and Phonology}
		{University of York, UK}
		{}
	\entry
		{2011 -- 2013}
		{ResMA in Theoretical and Applied Linguistics}
		{University of Pavia, IT}
		{}
	\entry
		{2007 -- 2011}
		{BA in Humanities (with Linguistics and Oriental Studies)}
		{University of Milan, IT}
		{}
\end{entrylist}

%----------------------------------------------------------------------------------------
%	PAPERS
%----------------------------------------------------------------------------------------

\cvsect{Papers}

\begin{entrylist}
	\entry
		{forthcoming}
		{Planting the seed for sound change: Evidence from real-time MRI of velum kinematics in German. \textnormal{\textit{Language}.} \vspace{0.5em}}
		{}
		{\small Chris Carignan, \underline{Stefano Coretta}, Jans Frahm, Jonathan Harrington, Phil Hoole, Arun Joseph, Esther Kunay, Dirk Voit}
	\entry
		{2021}
		{A cross cultural analysis of early prelinguistic gesture development and its relationship to language development. \textnormal{\textit{Child Development}. 92(1). 273--290. DOI: 10.1111/cdev.13406} \vspace{0.5em}}
		{}
		{\small Thea Cameron--Faulkner, Nivedita Malik, Circle Steele, \underline{Stefano Coretta}, Ludovica Serratrice, Elena Lieven}
	\entry
		{2020}
		{Longer vowel duration correlates with greater tongue root advancement at vowel offset: Acoustic and articulatory data from Italian and Polish. \textnormal{\textit{Journal of the Acoustical Society of America}. 147. 245--259. DOI: 10.1121/10.0000556} \vspace{0.5em}}
		{}
		{\small Stefano Coretta}
	\entry
		{2019}
		{An exploratory study of voicing‐related differences in vowel duration as compensatory temporal adjustment in Italian and Polish. \textnormal{\textit{Glossa: a journal of general linguistics}. 4(1), 125. DOI: 10.5334/gjgl.869} \vspace{0.5em}}
		{}
		{\small Stefano Coretta}
\end{entrylist}

%----------------------------------------------------------------------------------------
%	SOFTWARE
%----------------------------------------------------------------------------------------

\cvsect{Software}

\begin{entrylist}
	\entry
		{\hspace{1em}}
		{\textnormal{I develop and maintain open software aimed at facilitating linguistic research, all of which is hosted on GitHub and released under permissive (open) licenses (\url{https://github.com/stefanocoretta?tab=repositories}).}}
		{}
		{}
\end{entrylist}

%----------------------------------------------------------------------------------------
%	DATA
%----------------------------------------------------------------------------------------

\cvsect{Data}

\begin{entrylist}
	\entry
		{\hspace{1em}}
		{\textnormal{Research data from most of my solo and collaborative work can be found on the Open Science Framework (\url{https://osf.io/e84dv/}).}}
		{}
		{}
\end{entrylist}

%----------------------------------------------------------------------------------------
%	TEACHING
%----------------------------------------------------------------------------------------

\cvsect{Teaching}

\begin{entrylist}
	\entry
		{2018}
		{Experimental Phonetics \textnormal{(Teaching Assistant, \textit{University of Manchester, UK})}}
		{}
		{}
	\entry
		{2017}
		{Stylistics of English \textnormal{(Teaching Assistant, \textit{University of Manchester, UK})}}
		{}
		{}
	\entry
		{2016}
		{Forensic Linguistics of English \textnormal{(Teaching Assistant, \textit{University of Manchester, UK})}}
		{}
		{}
	\entry
		{2014}
		{Introduction to Phonology and Morphology \textnormal{(Teaching Assistant, \textit{University of Newcastle, Australia})}}
		{}
		{}
	\entry
		{2013}
		{Historical Linguistics \textnormal{(Teaching Assistant, \textit{University of Pavia, Italy})}}
		{}
		{}
\end{entrylist}

%----------------------------------------------------------------------------------------
%	RESEARCH APPOINTMENTS
%----------------------------------------------------------------------------------------

\cvsect{Research appointments}

\begin{entrylist}
	\entry
		{2018--2019}
		{\textnormal{Research Assistant for \textit{Investigating prelinguistic development in three minority cultures}, LuCiD.}}
		{}
		{Thea Cameron-Faulkner (PI), \url{http://lucid.ac.uk}}
	\entry
		{2018}
		{\textnormal{Research Assistant for \textit{Beauty is in the ear of the beholder: The role of accent in mate choice}.}}
		{}
		{Tucker Gilman (PI), University of Manchester, UK}
\end{entrylist}

%----------------------------------------------------------------------------------------
%	Professional service
%----------------------------------------------------------------------------------------

\cvsect{Professional service}

\begin{entrylist}
	\entry
		{Journal peer \\ review}
		{\textnormal{Journal of Phonetics, Journal of the Association for Laboratory Phonology, The Linguistic Review, Second Language Research.}}
		{}
		{}
	\entry
		{Event organisation}
		{\textnormal{Analysing Curves 2018 (UK), Manchester Forum in Linguistics 2017 and 2018, 25th Manchester Phonology Meeting 2017, Postgraduate Academic Researchers in Linguistics at York 2016.}}
		{}
		{}
\end{entrylist}

%----------------------------------------------------------------------------------------
%	ADDITIONAL INFORMATION
%----------------------------------------------------------------------------------------

\begin{minipage}[t]{0.3\textwidth}
	\vspace{-\baselineskip} % Required for vertically aligning minipages

	\cvsect{Languages}

	\textbf{Italian} - native\\
	\textbf{English} - C2 \\
	\textbf{Spanish} - A1
\end{minipage}
\begin{minipage}[t]{0.6\textwidth}
	\vspace{-\baselineskip} % Required for vertically aligning minipages

	\cvsect{Research skills}

	Research Methods, Statistics, Data Curation, Research Project Management, Version Control (git, dvc), Programming (R, Praat, Unix CLT, AWK, Matlab, Python, SQL), Markup (Rmarkdown, \LaTeX{}, HTML/CSS).
\end{minipage}


%----------------------------------------------------------------------------------------

\end{document}
