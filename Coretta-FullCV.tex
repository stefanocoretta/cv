\documentclass{article}

\usepackage{hyperref}
\usepackage[a4paper, margin=1in]{geometry}

\begin{document}

{\Huge \textbf{Full CV: Stefano Coretta}}

\section{GENERAL}

\begin{enumerate}
\item Name: Stefano Coretta
\item School: School of Psychology, Philosophy and Language Sciences
\item College: College of Arts, Humanities and Social Sciences
\item Date of first appointment: 04 October 2021
\item \textbf{Employment}
	\begin{itemize}
	\item 2021--present, Senior Teaching Coordinator (Stats), University of Edinburgh
	\item March 2023 (medical leave)
	\item 2019--2021, Postdoc researcher, Ludwig-Maximilians-Universität München
	\end{itemize}
\item \textbf{Education}
	\begin{itemize}
	\item 2016--2020 PhD in Linguistics, University of Manchester
	\item 2015-2016 MA in Phonetics and Phonology, University of York (First class)
	\item 2011-2013 ResMA in Theoretical and Applied Linguistics (110/110 cum laude)
	\item 2007-2011 BA in Classics and Humanities with Linguistics and Oriental Studies (103/110)
	\end{itemize}
\end{enumerate}

\section{TEACHING \& EDUCATION PROVISION}

\begin{enumerate}
\item \textbf{Teaching experience}
	\begin{itemize}
	\item 2022-2023
		\begin{itemize}
		\item Guided Research Seminar (\textbf{Making a Language: Conlanging and linguistics typology}), UG and PG. The course included an innovative type of assessment in which students were asked to create a conlang and describe it in a grammar sketch according to standard practices in linguistics description.
		\item \textbf{Statistics and Quantitative Methods (Semester 1)}, UG and PG. The course introduced statistics to students following the New Statistics approach, which focuses on estimation and uncertainty in light of the recent debates around the robustness of traditional statistical approaches.
		\item \textbf{Statistics and Quantitative Methods (Semester 2)}, UG and PG. This course adopted a Bayesian approach to inference, which lends itself better to answering research questions than the classical frequentist approach by allowing researchers to embed prior knowledge and to quantify uncertainty.
		\end{itemize}
	\item 2021-2022
		\begin{itemize}
		\item Research Methods in Develpmental Linguistics, PG. For this class, I invited three early career researchers from other institutions (Barcellona, Munich, Liverpool) to showcase resaerch methods in developmental linguistics they used in their research.
		\end{itemize}
	\item 2013-2018
		\begin{itemize}
		\item Teaching Assistant for Experimental Phonetics, Stylistics of English, Forensic Linguistics of English at the University of Manchester (PG), Introduction to Phonology and Morphology, at the University of Newcastle (Australia, PG), Historical linguistics at the University of Pavia (PG).
		\end{itemize}
	\end{itemize}
\item \textbf{Taught PG students supervised}: 5 (co-supervisions).
\item \textbf{Research-led teaching}: The stats courses I teach include approaches and methods from the recent literature on the replicability/generalisability crises and on robust inferential approaches like the "Bayesian New Statistics".
\end{enumerate}

\section{RESEARCH}

\begin{enumerate}
\item \textbf{Research interest:}
	\begin{itemize}
		\item Phonetics and phonology description, historical linguistics, descriptive linguistics, quantitative methods, research methods, statistics.
	\end{itemize}
\item \textbf{Evidence of output and impact}
	\begin{itemize}
		\item I have published 2 solo and 6 co-authored papers in top-tier, peer-reviewed journals in the field of linguistics and phonetics. One paper has the potential for great impact on the field of phonetics due to its involvement of more than 200 researchers (part of the Many Speech Analysis project, where 48 teams where asked to independently answer one question using the same data set).
	\end{itemize}
\item \textbf{Research PG students supervised}: I have infromally supervised 10 PhD students in the Linguistics and English Language department of the University of Edinburgh as part of the statistical consultations. This involved helping the students refine their research questions and hypotheses, their data analysis plan and their analysis skills, based on their research project. I have supervised one visiting PhD student (one month).
\end{enumerate}

\section{KNOWLEDGE EXCHANGE AND IMPACT}

\begin{enumerate}
\item \textbf{Workshops (external)}
	\begin{itemize}
  \item 2022-23 Workshops on Bayesian linear models and GAMMs at 4 international institutions.
  \item 2022 Introduction to Generalised Additive Models, University of Reading.
	\item 2021 intRo: Data visualisation with R, \url{https://intro-rstats.github.io}, sponsored by Engage at Liverpool and Methods North West.
	\item 2021 learnB4SS: Learn Bayesian Analysis for Speech Sciences. \url{https://learnb4ss.github.io}, delivered with Timo Roettger and Joseph V. Casillas, supported by LabPhon.
	\item 2018 An introduction to GAM(M)s, \url{https://github.com/stefanocoretta/gamm-workshop}, at Aarhus University, Denmark.
	\end{itemize}
\item \textit{Statistics for Linguistics} Network (\url{https://s4ln.github.io}) (219 members).
\end{enumerate}

\section{ACADEMIC LEADERSHIP, MANAGEMENT AND CITIZENSHIP}

\begin{enumerate}
\item \textbf{Academic Leadership and Management experience}
	\begin{itemize}
	\item I coordinate the provision of statistics within the Linguistics and English Language department, which involves managing the statistics offer both at UG and PG level across different programmes. As part of this task, I train colleagues in the department to ensure they are up to date with current statistical practices.
	\end{itemize}
\item \textbf{Consultancies}
	\begin{itemize}
	\item On top of the internal statistical consultance service (for University of Edinburgh staff and students), I have delivered statistical consultancies to staff and students of national and international institutions (10 hours total).
  \item I have planned, designed and delivered statistical workshops for national and international institutions (7 workshops over 2021-23).
	\end{itemize}
\end{enumerate}

\section{EXTERNAL RECOGNITION}

\textbf{Invited talks:}

\begin{itemize}
\item 2023. Rethinking the IPA vowel quadrilateral. Department of Linguistics and Scandinavian Studies seminar, 26 May, University of Oslo, Norway.
\item 2022. Today’s coarticulation is tomorrow’s sound change: Insights from dynamic articulatory data. Old World Conference on Phonology, Understanding Sound Change, 27 January, Donostia-San Sebastián, Spain.
\item 2021. The more we learn the less we know: (Un)resolved questions on the voicing effect. Linguistics department talk series. University of Potsdam, 28 July, Germany.
\item 2021. Exploring pathways to contrastive vowel nasalisation. Or what today’s articulatory patterns can tell us about yesterday’s sound change. Linguistics department talk series. University of Lancaster, 25 May, UK.
\item 2018. Vowel duration differences before voiceless and voiced stops as compensatory temporal adjustment. Linguistics department talk series. Indiana University Bloomington, 16 October, USA.
\item 2018. Vowel duration as a function of consonant gestural timing in Italian and Polish: Evidence from acoustics, ultrasound tongue imaging, and electroglottography. Linguistics department talk series. Aarhus University, 18 September, Denmark.
\end{itemize}

\section{LIST OF PUBLICATIONS}

\subsection{Books}

\begin{itemize}
	\item Enkeleida Kapia, \underline{Stefano Coretta}, L. Buxheli, A. Omari. 2023. \textit{Syntactic Microvariation in Albanian Dialects: An atlas with digital data.} Academy of Albanological Sciences. Institute of Linguistics and Literature. Tiranë. 349 pages.
\end{itemize}

\subsection{Peer-reviewed papers (sole author)}

\begin{itemize}
\item Stefano Coretta. 2020. Longer vowel duration correlates with greater tongue root advancement at vowel offset: Acoustic and articulatory data from Italian and Polish. \textit{Journal of the Acoustical Society of America}. 147. 245--259. DOI: 10.1121/10.0000556

\item Stefano Coretta. 2019 An exploratory study of voicing-related differences in vowel duration as compensatory temporal adjustment in Italian and Polish. \textit{Glossa: a journal of general linguistics}. 4(1), 125. DOI: 10.5334/gjgl.869
\end{itemize}

\subsection{Peer-reviewed papers (co-author)}

\begin{itemize}
\item \underline{Stefano Coretta}, Joseph V. Casillas, ..., Timo Roettger. 2023. Multidimensional signals and analytic flexibility: Estimating degrees of freedom in human speech analyses.\textit{Advances in Methods and Practices in Psychological Science}. 6(3). DOI: 10.1177/2515245923116256.

\item Bettelou Los, Dreschler Gea, Ans van Kemenade, Erwin Komen, \underline{Stefano Coretta}. 2023. The decline of local anchoring: A quantitative investigation. \textit{English Language \& Linguistics}. 27(2). 345--372.

\item \underline{Stefano Coretta}, Josiane Riverin-Coutlée, Enkeleida Kapia and Stephen Nichols. 2022. Northern Tosk Albanian (IPA Illustration). \textit{Journal of the International Phonetic Association}. 1--23. DOI: 10.1017/S0025100322000044

\item Patrycja Strycharczuk, Małgorzata Ćavar, \underline{Stefano Coretta}. 2021. Distance vs. time: Acoustic and articulatory consequences of reduced vowel duration in Polish. \textit{Journal of the Acoustical Society of America}. 150(1). 592--607. DOI: 10.1121/10.0005585.

\item Chris Carignan, \underline{Stefano Coretta}, Jans Frahm, Jonathan Harrington, Phil Hoole, Arun Joseph, Esther Kunay, Dirk Voit. 2021. Planting the seed for sound change: Evidence from real-time MRI of velum kinematics in German. \textit{Language}. 97(2). 333--364. \url{https://muse.jhu.edu/article/794876}.

\item Thea Cameron--Faulkner, Nivedita Malik, Circle Steele, \underline{Stefano Coretta}, Ludovica Serratrice, Elena Lieven. 2021. A cross cultural analysis of early prelinguistic gesture development and its relationship to language development. \textit{Child Development}. 92(1). 273--290. DOI: 10.1111/cdev.13406
\end{itemize}





\end{document}
